\documentclass[10pt,a4paper]{article}

\usepackage[utf8]{inputenc}
\usepackage{amsmath}
\usepackage{amsfonts}
\usepackage{amssymb}
\usepackage{graphicx}
\usepackage{listings}

\title{Execrcise sheet 8}
\author{Savu Victor-Gabriel}

\begin{document}
	\maketitle
	
	\section{Exercise 1}
	
	\begin{tabular}{l l}
		$code_V \, \text{let } (y_1, ..., y_n) = e_1 in e_0 \, \rho \; sd = $ &\\
		& $code_V \, e_1 \, \rho \; sd$ \\
		& for it = 1..m do \\
		& \hspace{5mm} get0 $j_{it}$ \\
		& pop \\
		& $code_V \, e_0 \, \rho \; sd + m$ \\
	\end{tabular}
	
	\subsection*{Optimization}
	
	We can make things simpler:
	
	get0 has 2 parameters j and sd
	
	the vector stays at the bottom of the stack
	
	\begin{tabular}{l l}
		$code_V \, \text{let } (y_1, ..., y_n) = e_1 in e_0 \, \rho \; sd = $ &\\
		& $code_V \, e_1 \, \rho \; sd$ \\
		& for it = 1..m do \\
		& \hspace{5mm} get0 $j_{it}$ $sd + j$ \\
		& $code_V \, e_0 \, \rho \; sd + m + 1$ \\
		& slide 1 \\
	\end{tabular}
	
	\section{Exercise 2}
	
	\subsection{Heap}
	
	We define a new heap type S
	
	(S,Type,Ref)
	
	Type is a value that represents the type of the reference Ref. If the type does not contain any data Ref is 0.
	
	\subsection{Instructions}
	
	We keep the tow instructions simple
	
	construct Type
	
	\begin{lstlisting}
		S[SP] = new (S, Type, S[SP])
	\end{lstlisting}
	
	deconstruct
	
	\begin{lstlisting}
		if ( S[SP] == (S,Type,Ref)) {
			S[SP+1] = Ref;
			S[SP+2] = Type;
			SP+=2;
		}
	\end{lstlisting}
	
	\subsection{CodeGen}
	
	\begin{lstlisting}[mathescape]
		mkvec 0
		mkfunval F
		jump callF

	F:
		deconstruct
		dup
		cmp \psi(A)
		jumpz caseB
		
	caseA:
		pop
		getbasic
		jump doneMatch
	
	caseB:
		dup
		cmp \psi(B)
		jumpz caseC
		pop
		getvector
		
		
		mark 2
		pushglob 0
		pushloc 1
		apply
	1:
		mark 2
		pushglob 0
		pushloc 3
		apply
	2:	
		sum
		slide 2
		jumpz doneMatch
		
	caseC:
		cmp \psi(C)
		jumpz explode
		
		pushc 0
		
	doneMatch:
		return 0
	
	explode:
		error "unmatched case"
		
	callF:
		mark end
		
		pushc 4
		pushc 0
		construct \psi(C)
		mkvec 2
		construct \psi(A)
		construct \psi(B)
		
		pushloc 1
		apply
		
	end:
		slide 1
	\end{lstlisting}
	
\end{document}